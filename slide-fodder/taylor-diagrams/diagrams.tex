\documentclass[12pt]{article}
\usepackage{sammath}

\newcommand{\half}{\frac{1}{2}}
\newcommand{\sixth}{\frac{1}{6}}
\newcommand{\dia}[1]{\includegraphics[height=0.7cm]{#1}}

\begin{document}

    \customsection{Diagrams for SGD Test Loss}
        We use diagrams for book-keeping of the Taylor terms (of test loss at a fixed initial weight). 
        Each color in a diagram represents a data value and thus selects a particular loss function from the
        data-parameterized distribution of loss functions.  Each node in a diagram represents a tensor obtained from
        derivatives of the loss function corresponding to that node's color.  It is a diagram's edges in a diagram that
        specify those derivative tensors.  We understand the edges as directed from left to right, with the source
        acting on the target by differentiation.  Finally, a diagram's value is the expectation over all iid assigments
        of data to the colors.  Thus,
        $$
            \dia{rg} := \expec_{\text{red},\text{green}} \nabla^a(l_{\text{red}}) \nabla^a(l_{\text{green}})
            ~~~~~
            \dia{rr} := \expec_{\text{red}} \nabla^a(l_{\text{red}}) \nabla^a(l_{\text{red}})
        $$
        $$
            \dia{rgb} := \expec_{\text{red},\text{green},\text{blue}} \nabla^a(l_{\text{red}}) \nabla^a \nabla^b(l_{\text{green}}) \nabla^b(l_{\text{blue}})
        $$
        $$
            \dia{sgd-2b} := \expec_{\text{red},\text{green},\text{blue}} \nabla^a(l_{\text{red}}) \nabla^b(l_{\text{green}}) \nabla^a \nabla^b(l_{\text{blue}})
        $$
        We see that $\dia{rr}-\dia{rg}$ gives the trace of the covariance of gradients.
        Moreover, $\dia{sgd-2a} = \dia{sgd-2b}$, illustrating how diagram notation can streamline computation by
        helping to group terms.  However, we caution that a diagram's value generally depends on a diagram's digraph 
        structure, not just its undirected structure.  For example:  
        $$
            \dia{rggb} = \dia{rggb-a} + \dia{rggb-b} \neq \dia{rrgb-a} = \dia{rrgb} 
        $$

        Thus prepared, we may expand the test loss after $T$ updates, each with batch-size $1$ sampled without
        replacement.  The recipe is to draw all the diagrams whose underlying poset has a unique rightmost element.
        Each node in the diagram contributes a symmetry factor $1/i!$ where $i$ is the node's in-degree.  On top of
        that, a diagram with $a$ edges and $v$ vertices has an overall combinatorial weight of
        $(-\eta)^a {T \choose v-1}$.  We obtain:
        \begin{align*}
            \expec \wrap{\text{SGD Test Loss}} = \dia{sgd-0}
            &- \eta   {T \choose 1} \wrap{\dia{sgd-1}} \\
            &+ \eta^2 {T \choose 2} \wrap{\dia{sgd-2a} + \half \dia{sgd-2b}}
             + \eta^2 {T \choose 1} \wrap{\half \dia{sgd-2c}} \\
            &- \eta^3 {T \choose 3} \wrap{\substack{
                    \dia{sgd-3a} + \half \dia{sgd-3b} + \half \dia{sgd-3c} + \\
                    \half \dia{sgd-3d} + \half \dia{sgd-3e} + \sixth \dia{sgd-3f}
                    }} \\  
            &- \eta^3 {T \choose 2} \wrap{\substack{
                    \half \dia{sgd-3g} + \sixth \dia{sgd-3i} + \\
                    \half \dia{sgd-3h} + \frac{2}{2} \dia{sgd-3k} + \sixth \dia{sgd-3j}
                    }} 
             - \eta^3 {T \choose 1} \wrap{\sixth \dia{sgd-3l}} 
             + o(\eta^3)
        \end{align*}
        And a routine grouping of terms yields:
        \begin{align*}
            \cdots = \dia{sgd-0}
            &- \eta   {T \choose 1} \wrap{\dia{sgd-1}} \\
            &+ \eta^2 {T \choose 2} \wrap{\frac{3}{2}\dia{sgd-2a}}
             + \eta^2 {T \choose 1} \wrap{\half \dia{sgd-2c}} \\
            &- \eta^3 {T \choose 3} \wrap{\frac{5}{2} \dia{sgd-3a} + \half \dia{sgd-3b} + \sixth \dia{sgd-3f}} \\
            &- \eta^3 {T \choose 2} \wrap{\dia{sgd-3g} + \frac{5}{6} \dia{sgd-3h} + \dia{sgd-3k}} 
             - \eta^3 {T \choose 1} \wrap{\sixth \dia{sgd-3l}} 
             + o(\eta^3)
        \end{align*}

\end{document}
